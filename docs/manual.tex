\documentclass[a4paper,notitlepage,11pt]{article}

%packages
\usepackage[utf8]{inputenc} %encodage en utf-8 pour la console UNIX
\usepackage[french]{babel}
\usepackage{latex2man} %importation du package permettant de générer un document LaTeX en manpage

%options
\pagestyle{plain} %imprime le num de la page

\begin{document}

\section{NOM}
./main.py - générateur de playlist personnalisée

\section{SYNOPSIS}
main.py nom {xspf,m3u,pls} temps [OPTIONS]

\section{DESCRIPTION}
Ce générateur de playlist permet de générer un fichier dans un format spécial qui contient un ensemble de titres musicaux selon les choix de l'utilisateur.

Voici l'utilisation basique du programme (sans options avancées), l'ensemble des options listées sont obligatoire pour générer la playlist.
\begin{description}
\item[commande -] il s'agit du nom du fichier qui sera exécuté pour utiliser le programme (".\textbackslash{}main.py")
\item[nom -] nom du fichier de sortie (inutile de préciser l'extension, elle sera ajouter automatiquement selon l'option suivante)
\item[format -] ajoute l'extension au fichier selon 3 choix possible (xspf, m3u ou pls)
\item[temps -] permet de spécifier la durée totale de la playlist en minute
\item['-h' '--help' '--aide' :] affiche l'aide
\item['-v' '--verbeux' :] permet l'affichage détaillé des opérations (ces informations sont également disponible dans le fichier de log)
note : le mode verbeux n'affiche pas grand chose s'il est envoyé seul.
\end{description}

\section{EXEMPLES}
\begin{description}
\item[".\textbackslash{}main.py play m3u 150" -] générera le fichier "play.m3u" qui contiendra une playlist d'une durée de 150 minutes.
\item[".\textbackslash{}main.py -h" -] affiche l'aide (sans exécuter la génération de la playlist).
\item[".\textbackslash{}main.py play m3u 150 --verbeux" -] affiche un ensemble d'information (particulièrement utiles pour les développeurs)
\end{description}

\section{OPTIONS}


\section{FICHIERS}
\begin{description}
Le packet contient les fichiers suivants:
\item[main.py -] programme principal
\item[config.py -] fichier de configuration
\item[playlist.log -] fichier de log (se remplit à l'utilisation)
\item[docs/manual.pdf -] le présent manuel
\end{description}

\section{REQUIS}
L'utilisation du générateur exige l'installation des applications suivantes :
\begin{itemize}
\item python3
\item python-pycogs
\item python3-sqlalchemy
\end{itemize}

\section{TRADUCTION}
français

\section{VERSION}
version 1.0b

\section{LICENCE}
GPL v3

\section{AUTEURS}

\end{document}
